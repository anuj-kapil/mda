\documentclass[]{article}
\usepackage{lmodern}
\usepackage{amssymb,amsmath}
\usepackage{ifxetex,ifluatex}
\usepackage{fixltx2e} % provides \textsubscript
\ifnum 0\ifxetex 1\fi\ifluatex 1\fi=0 % if pdftex
  \usepackage[T1]{fontenc}
  \usepackage[utf8]{inputenc}
\else % if luatex or xelatex
  \ifxetex
    \usepackage{mathspec}
  \else
    \usepackage{fontspec}
  \fi
  \defaultfontfeatures{Ligatures=TeX,Scale=MatchLowercase}
\fi
% use upquote if available, for straight quotes in verbatim environments
\IfFileExists{upquote.sty}{\usepackage{upquote}}{}
% use microtype if available
\IfFileExists{microtype.sty}{%
\usepackage{microtype}
\UseMicrotypeSet[protrusion]{basicmath} % disable protrusion for tt fonts
}{}
\usepackage[margin=1in]{geometry}
\usepackage{hyperref}
\hypersetup{unicode=true,
            pdftitle={Multivariate Data Analysis - Presentation},
            pdfborder={0 0 0},
            breaklinks=true}
\urlstyle{same}  % don't use monospace font for urls
\usepackage{color}
\usepackage{fancyvrb}
\newcommand{\VerbBar}{|}
\newcommand{\VERB}{\Verb[commandchars=\\\{\}]}
\DefineVerbatimEnvironment{Highlighting}{Verbatim}{commandchars=\\\{\}}
% Add ',fontsize=\small' for more characters per line
\usepackage{framed}
\definecolor{shadecolor}{RGB}{248,248,248}
\newenvironment{Shaded}{\begin{snugshade}}{\end{snugshade}}
\newcommand{\AlertTok}[1]{\textcolor[rgb]{0.94,0.16,0.16}{#1}}
\newcommand{\AnnotationTok}[1]{\textcolor[rgb]{0.56,0.35,0.01}{\textbf{\textit{#1}}}}
\newcommand{\AttributeTok}[1]{\textcolor[rgb]{0.77,0.63,0.00}{#1}}
\newcommand{\BaseNTok}[1]{\textcolor[rgb]{0.00,0.00,0.81}{#1}}
\newcommand{\BuiltInTok}[1]{#1}
\newcommand{\CharTok}[1]{\textcolor[rgb]{0.31,0.60,0.02}{#1}}
\newcommand{\CommentTok}[1]{\textcolor[rgb]{0.56,0.35,0.01}{\textit{#1}}}
\newcommand{\CommentVarTok}[1]{\textcolor[rgb]{0.56,0.35,0.01}{\textbf{\textit{#1}}}}
\newcommand{\ConstantTok}[1]{\textcolor[rgb]{0.00,0.00,0.00}{#1}}
\newcommand{\ControlFlowTok}[1]{\textcolor[rgb]{0.13,0.29,0.53}{\textbf{#1}}}
\newcommand{\DataTypeTok}[1]{\textcolor[rgb]{0.13,0.29,0.53}{#1}}
\newcommand{\DecValTok}[1]{\textcolor[rgb]{0.00,0.00,0.81}{#1}}
\newcommand{\DocumentationTok}[1]{\textcolor[rgb]{0.56,0.35,0.01}{\textbf{\textit{#1}}}}
\newcommand{\ErrorTok}[1]{\textcolor[rgb]{0.64,0.00,0.00}{\textbf{#1}}}
\newcommand{\ExtensionTok}[1]{#1}
\newcommand{\FloatTok}[1]{\textcolor[rgb]{0.00,0.00,0.81}{#1}}
\newcommand{\FunctionTok}[1]{\textcolor[rgb]{0.00,0.00,0.00}{#1}}
\newcommand{\ImportTok}[1]{#1}
\newcommand{\InformationTok}[1]{\textcolor[rgb]{0.56,0.35,0.01}{\textbf{\textit{#1}}}}
\newcommand{\KeywordTok}[1]{\textcolor[rgb]{0.13,0.29,0.53}{\textbf{#1}}}
\newcommand{\NormalTok}[1]{#1}
\newcommand{\OperatorTok}[1]{\textcolor[rgb]{0.81,0.36,0.00}{\textbf{#1}}}
\newcommand{\OtherTok}[1]{\textcolor[rgb]{0.56,0.35,0.01}{#1}}
\newcommand{\PreprocessorTok}[1]{\textcolor[rgb]{0.56,0.35,0.01}{\textit{#1}}}
\newcommand{\RegionMarkerTok}[1]{#1}
\newcommand{\SpecialCharTok}[1]{\textcolor[rgb]{0.00,0.00,0.00}{#1}}
\newcommand{\SpecialStringTok}[1]{\textcolor[rgb]{0.31,0.60,0.02}{#1}}
\newcommand{\StringTok}[1]{\textcolor[rgb]{0.31,0.60,0.02}{#1}}
\newcommand{\VariableTok}[1]{\textcolor[rgb]{0.00,0.00,0.00}{#1}}
\newcommand{\VerbatimStringTok}[1]{\textcolor[rgb]{0.31,0.60,0.02}{#1}}
\newcommand{\WarningTok}[1]{\textcolor[rgb]{0.56,0.35,0.01}{\textbf{\textit{#1}}}}
\usepackage{graphicx,grffile}
\makeatletter
\def\maxwidth{\ifdim\Gin@nat@width>\linewidth\linewidth\else\Gin@nat@width\fi}
\def\maxheight{\ifdim\Gin@nat@height>\textheight\textheight\else\Gin@nat@height\fi}
\makeatother
% Scale images if necessary, so that they will not overflow the page
% margins by default, and it is still possible to overwrite the defaults
% using explicit options in \includegraphics[width, height, ...]{}
\setkeys{Gin}{width=\maxwidth,height=\maxheight,keepaspectratio}
\IfFileExists{parskip.sty}{%
\usepackage{parskip}
}{% else
\setlength{\parindent}{0pt}
\setlength{\parskip}{6pt plus 2pt minus 1pt}
}
\setlength{\emergencystretch}{3em}  % prevent overfull lines
\providecommand{\tightlist}{%
  \setlength{\itemsep}{0pt}\setlength{\parskip}{0pt}}
\setcounter{secnumdepth}{0}
% Redefines (sub)paragraphs to behave more like sections
\ifx\paragraph\undefined\else
\let\oldparagraph\paragraph
\renewcommand{\paragraph}[1]{\oldparagraph{#1}\mbox{}}
\fi
\ifx\subparagraph\undefined\else
\let\oldsubparagraph\subparagraph
\renewcommand{\subparagraph}[1]{\oldsubparagraph{#1}\mbox{}}
\fi

%%% Use protect on footnotes to avoid problems with footnotes in titles
\let\rmarkdownfootnote\footnote%
\def\footnote{\protect\rmarkdownfootnote}

%%% Change title format to be more compact
\usepackage{titling}

% Create subtitle command for use in maketitle
\providecommand{\subtitle}[1]{
  \posttitle{
    \begin{center}\large#1\end{center}
    }
}

\setlength{\droptitle}{-2em}

  \title{Multivariate Data Analysis - Presentation}
    \pretitle{\vspace{\droptitle}\centering\huge}
  \posttitle{\par}
    \author{}
    \preauthor{}\postauthor{}
    \date{}
    \predate{}\postdate{}
  

\begin{document}
\maketitle

\hypertarget{multivariate-data-analysis-spring-2019-37459-2019-spring-city}{%
\subsection{Multivariate Data Analysis Spring 2019
(37459-2019-SPRING-CITY)}\label{multivariate-data-analysis-spring-2019-37459-2019-spring-city}}

\hypertarget{assignment-presentation}{%
\subsubsection{Assignment: Presentation}\label{assignment-presentation}}

\hypertarget{student-name-anuj-kapil}{%
\subsubsection{Student Name: Anuj Kapil}\label{student-name-anuj-kapil}}

\hypertarget{student-id-12678708}{%
\subsubsection{Student Id: 12678708}\label{student-id-12678708}}

\hypertarget{step-1-download-the-dataset-from-kaggle}{%
\subsection{Step 1: Download the dataset from
Kaggle}\label{step-1-download-the-dataset-from-kaggle}}

To simplify the process, I am using my github account to host the
dataset but the original dataset is available here:
\href{https://www.kaggle.com/harlfoxem/housesalesprediction}{Kaggle:
House Sales in King County}

\begin{Shaded}
\begin{Highlighting}[]
\NormalTok{wd <-}\StringTok{ }\KeywordTok{getwd}\NormalTok{()}
\NormalTok{filep <-}\StringTok{ "https://raw.githubusercontent.com/anuj-kapil/mda/master/Data/kc_house_data.csv"}
\NormalTok{filename <-}\StringTok{ "kc_house_data.csv"}
\KeywordTok{download.file}\NormalTok{(}\DataTypeTok{url=}\NormalTok{filep, }\DataTypeTok{destfile=}\NormalTok{filename)}
\end{Highlighting}
\end{Shaded}

\hypertarget{step-2-read-the-downloaded-file-to-memory}{%
\subsection{Step 2: Read the downloaded file to
memory}\label{step-2-read-the-downloaded-file-to-memory}}

We are using the data.table package from R to read the file in to memory

\begin{Shaded}
\begin{Highlighting}[]
\NormalTok{kc_houses<-}\KeywordTok{fread}\NormalTok{(}\StringTok{'kc_house_data.csv'}\NormalTok{)}
\end{Highlighting}
\end{Shaded}

\hypertarget{step-3-exploratory-analysis}{%
\subsection{Step 3: Exploratory
Analysis}\label{step-3-exploratory-analysis}}

Let's have a look at the number of features and observations in the
dataset

\begin{Shaded}
\begin{Highlighting}[]
\CommentTok{# Rows and columns}
\KeywordTok{dim}\NormalTok{(kc_houses)}
\end{Highlighting}
\end{Shaded}

\begin{verbatim}
## [1] 21597    21
\end{verbatim}

Quick look at the data types and ranges of data. Below summary shows
that there are no null or missing values

\begin{Shaded}
\begin{Highlighting}[]
\CommentTok{# No missing values}
\KeywordTok{summary}\NormalTok{(kc_houses)}
\end{Highlighting}
\end{Shaded}

\begin{verbatim}
##        id                 date               price        
##  Min.   :   1000102   Length:21597       Min.   :  78000  
##  1st Qu.:2123049175   Class :character   1st Qu.: 322000  
##  Median :3904930410   Mode  :character   Median : 450000  
##  Mean   :4580474287                      Mean   : 540297  
##  3rd Qu.:7308900490                      3rd Qu.: 645000  
##  Max.   :9900000190                      Max.   :7700000  
##     bedrooms        bathrooms      sqft_living       sqft_lot      
##  Min.   : 1.000   Min.   :0.500   Min.   :  370   Min.   :    520  
##  1st Qu.: 3.000   1st Qu.:1.750   1st Qu.: 1430   1st Qu.:   5040  
##  Median : 3.000   Median :2.250   Median : 1910   Median :   7618  
##  Mean   : 3.373   Mean   :2.116   Mean   : 2080   Mean   :  15099  
##  3rd Qu.: 4.000   3rd Qu.:2.500   3rd Qu.: 2550   3rd Qu.:  10685  
##  Max.   :33.000   Max.   :8.000   Max.   :13540   Max.   :1651359  
##      floors        waterfront            view          condition   
##  Min.   :1.000   Min.   :0.000000   Min.   :0.0000   Min.   :1.00  
##  1st Qu.:1.000   1st Qu.:0.000000   1st Qu.:0.0000   1st Qu.:3.00  
##  Median :1.500   Median :0.000000   Median :0.0000   Median :3.00  
##  Mean   :1.494   Mean   :0.007547   Mean   :0.2343   Mean   :3.41  
##  3rd Qu.:2.000   3rd Qu.:0.000000   3rd Qu.:0.0000   3rd Qu.:4.00  
##  Max.   :3.500   Max.   :1.000000   Max.   :4.0000   Max.   :5.00  
##      grade          sqft_above   sqft_basement       yr_built   
##  Min.   : 3.000   Min.   : 370   Min.   :   0.0   Min.   :1900  
##  1st Qu.: 7.000   1st Qu.:1190   1st Qu.:   0.0   1st Qu.:1951  
##  Median : 7.000   Median :1560   Median :   0.0   Median :1975  
##  Mean   : 7.658   Mean   :1789   Mean   : 291.7   Mean   :1971  
##  3rd Qu.: 8.000   3rd Qu.:2210   3rd Qu.: 560.0   3rd Qu.:1997  
##  Max.   :13.000   Max.   :9410   Max.   :4820.0   Max.   :2015  
##   yr_renovated        zipcode           lat             long       
##  Min.   :   0.00   Min.   :98001   Min.   :47.16   Min.   :-122.5  
##  1st Qu.:   0.00   1st Qu.:98033   1st Qu.:47.47   1st Qu.:-122.3  
##  Median :   0.00   Median :98065   Median :47.57   Median :-122.2  
##  Mean   :  84.46   Mean   :98078   Mean   :47.56   Mean   :-122.2  
##  3rd Qu.:   0.00   3rd Qu.:98118   3rd Qu.:47.68   3rd Qu.:-122.1  
##  Max.   :2015.00   Max.   :98199   Max.   :47.78   Max.   :-121.3  
##  sqft_living15    sqft_lot15    
##  Min.   : 399   Min.   :   651  
##  1st Qu.:1490   1st Qu.:  5100  
##  Median :1840   Median :  7620  
##  Mean   :1987   Mean   : 12758  
##  3rd Qu.:2360   3rd Qu.: 10083  
##  Max.   :6210   Max.   :871200
\end{verbatim}

The dates are loaded as string of characters. Let's convert them to look
at the range of values. The summary now shows that the dates are in the
range of May 2014 - May 2015

\begin{Shaded}
\begin{Highlighting}[]
\CommentTok{# Convert Dates}
\NormalTok{kc_houses}\OperatorTok{$}\NormalTok{date<-}\KeywordTok{as.IDate}\NormalTok{(kc_houses}\OperatorTok{$}\NormalTok{date, }\DataTypeTok{format =} \StringTok{"%m/%d/%Y"}\NormalTok{)}
\KeywordTok{summary}\NormalTok{(kc_houses)}
\end{Highlighting}
\end{Shaded}

\begin{verbatim}
##        id                  date                price        
##  Min.   :   1000102   Min.   :2014-05-02   Min.   :  78000  
##  1st Qu.:2123049175   1st Qu.:2014-07-22   1st Qu.: 322000  
##  Median :3904930410   Median :2014-10-16   Median : 450000  
##  Mean   :4580474287   Mean   :2014-10-29   Mean   : 540297  
##  3rd Qu.:7308900490   3rd Qu.:2015-02-17   3rd Qu.: 645000  
##  Max.   :9900000190   Max.   :2015-05-27   Max.   :7700000  
##     bedrooms        bathrooms      sqft_living       sqft_lot      
##  Min.   : 1.000   Min.   :0.500   Min.   :  370   Min.   :    520  
##  1st Qu.: 3.000   1st Qu.:1.750   1st Qu.: 1430   1st Qu.:   5040  
##  Median : 3.000   Median :2.250   Median : 1910   Median :   7618  
##  Mean   : 3.373   Mean   :2.116   Mean   : 2080   Mean   :  15099  
##  3rd Qu.: 4.000   3rd Qu.:2.500   3rd Qu.: 2550   3rd Qu.:  10685  
##  Max.   :33.000   Max.   :8.000   Max.   :13540   Max.   :1651359  
##      floors        waterfront            view          condition   
##  Min.   :1.000   Min.   :0.000000   Min.   :0.0000   Min.   :1.00  
##  1st Qu.:1.000   1st Qu.:0.000000   1st Qu.:0.0000   1st Qu.:3.00  
##  Median :1.500   Median :0.000000   Median :0.0000   Median :3.00  
##  Mean   :1.494   Mean   :0.007547   Mean   :0.2343   Mean   :3.41  
##  3rd Qu.:2.000   3rd Qu.:0.000000   3rd Qu.:0.0000   3rd Qu.:4.00  
##  Max.   :3.500   Max.   :1.000000   Max.   :4.0000   Max.   :5.00  
##      grade          sqft_above   sqft_basement       yr_built   
##  Min.   : 3.000   Min.   : 370   Min.   :   0.0   Min.   :1900  
##  1st Qu.: 7.000   1st Qu.:1190   1st Qu.:   0.0   1st Qu.:1951  
##  Median : 7.000   Median :1560   Median :   0.0   Median :1975  
##  Mean   : 7.658   Mean   :1789   Mean   : 291.7   Mean   :1971  
##  3rd Qu.: 8.000   3rd Qu.:2210   3rd Qu.: 560.0   3rd Qu.:1997  
##  Max.   :13.000   Max.   :9410   Max.   :4820.0   Max.   :2015  
##   yr_renovated        zipcode           lat             long       
##  Min.   :   0.00   Min.   :98001   Min.   :47.16   Min.   :-122.5  
##  1st Qu.:   0.00   1st Qu.:98033   1st Qu.:47.47   1st Qu.:-122.3  
##  Median :   0.00   Median :98065   Median :47.57   Median :-122.2  
##  Mean   :  84.46   Mean   :98078   Mean   :47.56   Mean   :-122.2  
##  3rd Qu.:   0.00   3rd Qu.:98118   3rd Qu.:47.68   3rd Qu.:-122.1  
##  Max.   :2015.00   Max.   :98199   Max.   :47.78   Max.   :-121.3  
##  sqft_living15    sqft_lot15    
##  Min.   : 399   Min.   :   651  
##  1st Qu.:1490   1st Qu.:  5100  
##  Median :1840   Median :  7620  
##  Mean   :1987   Mean   : 12758  
##  3rd Qu.:2360   3rd Qu.: 10083  
##  Max.   :6210   Max.   :871200
\end{verbatim}

\hypertarget{multicollinearity-analysis}{%
\subsection{Multicollinearity
Analysis}\label{multicollinearity-analysis}}

\begin{Shaded}
\begin{Highlighting}[]
\CommentTok{#Corr Plot}
\NormalTok{cor_mat <-}\StringTok{ }\NormalTok{kc_houses[,}\DecValTok{3}\OperatorTok{:}\DecValTok{21}\NormalTok{]}
\NormalTok{corr <-}\StringTok{ }\KeywordTok{cor}\NormalTok{(cor_mat, }\DataTypeTok{use =} \StringTok{"pairwise.complete.obs"}\NormalTok{)}

\KeywordTok{ggcorrplot}\NormalTok{(corr, }\DataTypeTok{hc.order =} \OtherTok{FALSE}\NormalTok{, }\DataTypeTok{type =} \StringTok{"lower"}\NormalTok{,}
           \DataTypeTok{ggtheme =}\NormalTok{ ggthemes}\OperatorTok{::}\NormalTok{theme_gdocs,}
           \DataTypeTok{colors =} \KeywordTok{c}\NormalTok{(}\StringTok{"#ff7f0e"}\NormalTok{, }\StringTok{"white"}\NormalTok{, }\StringTok{"#1f83b4"}\NormalTok{),}
           \DataTypeTok{lab =} \OtherTok{TRUE}\NormalTok{)}\OperatorTok{+}
\StringTok{           }\KeywordTok{theme}\NormalTok{(}\DataTypeTok{panel.grid.major=}\KeywordTok{element_blank}\NormalTok{())}
\end{Highlighting}
\end{Shaded}

\includegraphics{ClassPresentation_files/figure-latex/unnamed-chunk-6-1.pdf}

\hypertarget{univariate-analysis}{%
\subsection{Univariate Analysis}\label{univariate-analysis}}

In case of just the dependent variable and 0 independent variable, the
best fit model is the mean of the dependent variable.

\begin{Shaded}
\begin{Highlighting}[]
\CommentTok{# Univariate model}
\KeywordTok{ggplot}\NormalTok{(kc_houses, }\KeywordTok{aes}\NormalTok{(}\DataTypeTok{x=}\KeywordTok{seq_along}\NormalTok{(id), }\DataTypeTok{y=}\NormalTok{price))}\OperatorTok{+}
\StringTok{  }\KeywordTok{geom_point}\NormalTok{(}\DataTypeTok{col =} \StringTok{"#1f83b4"}\NormalTok{)}\OperatorTok{+}
\StringTok{  }\KeywordTok{geom_hline}\NormalTok{(}\KeywordTok{aes}\NormalTok{(}\DataTypeTok{yintercept =} \KeywordTok{mean}\NormalTok{(kc_houses}\OperatorTok{$}\NormalTok{price,}\DataTypeTok{na.rm =}\NormalTok{ T)), }\DataTypeTok{col =} \StringTok{"#ff7f0e"}\NormalTok{)}\OperatorTok{+}\StringTok{ }
\StringTok{  }\KeywordTok{scale_color_tableau}\NormalTok{() }\OperatorTok{+}
\StringTok{  }\KeywordTok{labs}\NormalTok{(}\DataTypeTok{x=}\StringTok{"Id"}\NormalTok{, }\DataTypeTok{y=}\StringTok{"Price"}\NormalTok{)}\OperatorTok{+}
\StringTok{  }\KeywordTok{theme}\NormalTok{(}\DataTypeTok{panel.background =} \KeywordTok{element_blank}\NormalTok{(), }\DataTypeTok{axis.line =} \KeywordTok{element_line}\NormalTok{(}\DataTypeTok{colour =} \StringTok{"grey"}\NormalTok{), }\DataTypeTok{plot.title =} \KeywordTok{element_text}\NormalTok{(}\DataTypeTok{hjust =} \FloatTok{0.5}\NormalTok{))}
\end{Highlighting}
\end{Shaded}

\includegraphics{ClassPresentation_files/figure-latex/unnamed-chunk-7-1.pdf}

Calculating the Sum of Square of Errors/Residuals

\begin{Shaded}
\begin{Highlighting}[]
\NormalTok{kc_houses[,err}\OperatorTok{:}\ErrorTok{=}\NormalTok{price}\OperatorTok{-}\KeywordTok{mean}\NormalTok{(price,}\DataTypeTok{na.rm =}\NormalTok{ T)]}
\NormalTok{kc_houses[,err_sq}\OperatorTok{:}\ErrorTok{=}\NormalTok{err}\OperatorTok{^}\DecValTok{2}\NormalTok{]}
\NormalTok{sse <-}\StringTok{ }\KeywordTok{sum}\NormalTok{(kc_houses}\OperatorTok{$}\NormalTok{err_sq)}
\KeywordTok{options}\NormalTok{(}\DataTypeTok{scipen =} \DecValTok{999}\NormalTok{)}
\CommentTok{# Total Sum of Squares = Sum of Square of Residuals}
\NormalTok{sse}
\end{Highlighting}
\end{Shaded}

\begin{verbatim}
## [1] 2914582130408202
\end{verbatim}

\begin{Shaded}
\begin{Highlighting}[]
\CommentTok{# Sample mean}
\KeywordTok{mean}\NormalTok{(kc_houses}\OperatorTok{$}\NormalTok{price,}\DataTypeTok{na.rm =}\NormalTok{ T)}
\end{Highlighting}
\end{Shaded}

\begin{verbatim}
## [1] 540296.6
\end{verbatim}

\hypertarget{simple-linear-regression}{%
\subsection{Simple Linear Regression}\label{simple-linear-regression}}

Let us look at scatter plots of dependent variables and 3 chosen
independent variables. Below is the function created to combine the
scatter plot, histograms and density plots.

\begin{Shaded}
\begin{Highlighting}[]
\NormalTok{ggplot_scatter_hist_combo <-}\StringTok{ }\ControlFlowTok{function}\NormalTok{(dat, a, b, lab_a, lab_b)\{}
  \CommentTok{# if(is.null(lab_a))}
  \CommentTok{#   lab_a<-a}
  \CommentTok{# if(is.null(lab_b))}
  \CommentTok{#   lab_b<-b}

\NormalTok{  theme_blank<-}\StringTok{ }\KeywordTok{theme}\NormalTok{(}\DataTypeTok{axis.line=}\KeywordTok{element_blank}\NormalTok{(),}\DataTypeTok{axis.text.x=}\KeywordTok{element_blank}\NormalTok{(),}
                      \DataTypeTok{axis.text.y=}\KeywordTok{element_blank}\NormalTok{(),}\DataTypeTok{axis.ticks=}\KeywordTok{element_blank}\NormalTok{(),}
                      \DataTypeTok{axis.title.x=}\KeywordTok{element_blank}\NormalTok{(),}
                      \DataTypeTok{axis.title.y=}\KeywordTok{element_blank}\NormalTok{(),}\DataTypeTok{legend.position=}\StringTok{"none"}\NormalTok{,}
                      \DataTypeTok{panel.background=}\KeywordTok{element_rect}\NormalTok{(}\DataTypeTok{fill =} \StringTok{"transparent"}\NormalTok{),}\DataTypeTok{panel.border=}\KeywordTok{element_blank}\NormalTok{(),}\DataTypeTok{panel.grid.major=}\KeywordTok{element_blank}\NormalTok{(),}
                      \DataTypeTok{panel.grid.minor=}\KeywordTok{element_blank}\NormalTok{(),}\DataTypeTok{plot.background=}\KeywordTok{element_rect}\NormalTok{(}\DataTypeTok{fill =} \StringTok{"transparent"}\NormalTok{, }\DataTypeTok{color =} \OtherTok{NA}\NormalTok{))}
  
\NormalTok{  hist_top<-}\KeywordTok{ggplot}\NormalTok{(dat, }\KeywordTok{aes}\NormalTok{(}\DataTypeTok{x=}\NormalTok{a)) }\OperatorTok{+}\StringTok{ }
\StringTok{    }\KeywordTok{geom_histogram}\NormalTok{(}\KeywordTok{aes}\NormalTok{(}\DataTypeTok{y =}\NormalTok{..density..), }
                   \DataTypeTok{col=}\StringTok{"#ff7f0e"}\NormalTok{, }
                   \DataTypeTok{fill=}\StringTok{"#ffaa0e"}\NormalTok{, }
                   \DataTypeTok{alpha =} \FloatTok{.2}\NormalTok{,}
                   \DataTypeTok{bins =} \DecValTok{35}\NormalTok{) }\OperatorTok{+}\StringTok{ }
\StringTok{    }\KeywordTok{geom_density}\NormalTok{(}\DataTypeTok{col=}\StringTok{"#ff7f0e"}\NormalTok{) }\OperatorTok{+}\StringTok{ }
\StringTok{    }\KeywordTok{scale_color_tableau}\NormalTok{() }\OperatorTok{+}
\StringTok{    }\NormalTok{theme_blank}
  
\NormalTok{  empty <-}\StringTok{ }\KeywordTok{ggplot}\NormalTok{()}\OperatorTok{+}\KeywordTok{geom_point}\NormalTok{(}\KeywordTok{aes}\NormalTok{(}\DecValTok{1}\NormalTok{,}\DecValTok{1}\NormalTok{), }\DataTypeTok{colour=}\StringTok{"white"}\NormalTok{)}\OperatorTok{+}
\StringTok{    }\NormalTok{theme_blank}
  
\NormalTok{  scatter<-}\KeywordTok{ggplot}\NormalTok{(dat, }\KeywordTok{aes}\NormalTok{(}\DataTypeTok{x=}\NormalTok{a, }\DataTypeTok{y=}\NormalTok{b))}\OperatorTok{+}
\StringTok{    }\KeywordTok{geom_point}\NormalTok{(}\DataTypeTok{col=}\StringTok{"#1f83b4"}\NormalTok{)}\OperatorTok{+}
\StringTok{    }\KeywordTok{labs}\NormalTok{(}\DataTypeTok{x=}\NormalTok{lab_a, }\DataTypeTok{y=}\NormalTok{lab_b)}\OperatorTok{+}
\StringTok{    }\KeywordTok{theme}\NormalTok{(}\DataTypeTok{panel.background =} \KeywordTok{element_blank}\NormalTok{(), }\DataTypeTok{axis.line =} \KeywordTok{element_line}\NormalTok{(}\DataTypeTok{colour =} \StringTok{"grey"}\NormalTok{), }\DataTypeTok{plot.title =} \KeywordTok{element_text}\NormalTok{(}\DataTypeTok{hjust =} \FloatTok{0.5}\NormalTok{))}
  
\NormalTok{  hist_right <-}\StringTok{ }\KeywordTok{ggplot}\NormalTok{(dat, }\KeywordTok{aes}\NormalTok{(}\DataTypeTok{x=}\NormalTok{b)) }\OperatorTok{+}\StringTok{ }
\StringTok{    }\KeywordTok{geom_histogram}\NormalTok{(}\KeywordTok{aes}\NormalTok{(}\DataTypeTok{y =}\NormalTok{..density..), }
                   \DataTypeTok{col=}\StringTok{"#ff7f0e"}\NormalTok{, }
                   \DataTypeTok{fill=}\StringTok{"#ffaa0e"}\NormalTok{, }
                   \DataTypeTok{alpha =} \FloatTok{.2}\NormalTok{,}
                   \DataTypeTok{bins =} \DecValTok{35}\NormalTok{) }\OperatorTok{+}\StringTok{ }
\StringTok{    }\KeywordTok{geom_density}\NormalTok{(}\DataTypeTok{col=}\StringTok{"#ff7f0e"}\NormalTok{) }\OperatorTok{+}\StringTok{ }
\StringTok{    }\KeywordTok{coord_flip}\NormalTok{() }\OperatorTok{+}
\StringTok{    }\NormalTok{theme_blank}
  
  \KeywordTok{grid.arrange}\NormalTok{(hist_top, empty, scatter, hist_right, }\DataTypeTok{ncol=}\DecValTok{2}\NormalTok{, }\DataTypeTok{nrow=}\DecValTok{2}\NormalTok{, }\DataTypeTok{widths=}\KeywordTok{c}\NormalTok{(}\DecValTok{4}\NormalTok{, }\DecValTok{1}\NormalTok{), }\DataTypeTok{heights=}\KeywordTok{c}\NormalTok{(}\DecValTok{1}\NormalTok{, }\DecValTok{4}\NormalTok{))}
\NormalTok{\}}
\end{Highlighting}
\end{Shaded}

\hypertarget{scatterhistogramdensity-plots-sq.-ft-living-vs-price}{%
\subsubsection{Scatter/Histogram/Density Plots: Sq. Ft Living vs
Price}\label{scatterhistogramdensity-plots-sq.-ft-living-vs-price}}

\begin{Shaded}
\begin{Highlighting}[]
\KeywordTok{ggplot_scatter_hist_combo}\NormalTok{(}\DataTypeTok{dat =}\NormalTok{ kc_houses, }\DataTypeTok{a=}\NormalTok{kc_houses}\OperatorTok{$}\NormalTok{sqft_living, }\DataTypeTok{b=}\NormalTok{kc_houses}\OperatorTok{$}\NormalTok{price, }\DataTypeTok{lab_a =} \StringTok{"Sq. Ft. Living"}\NormalTok{, }\DataTypeTok{lab_b =} \StringTok{"Price"}\NormalTok{)}
\end{Highlighting}
\end{Shaded}

\includegraphics{ClassPresentation_files/figure-latex/unnamed-chunk-10-1.pdf}

\hypertarget{scatterhistogramdensity-plots-grade-vs-price}{%
\subsubsection{Scatter/Histogram/Density Plots: Grade vs
Price}\label{scatterhistogramdensity-plots-grade-vs-price}}

\begin{Shaded}
\begin{Highlighting}[]
\KeywordTok{ggplot_scatter_hist_combo}\NormalTok{(}\DataTypeTok{dat =}\NormalTok{ kc_houses, }\DataTypeTok{a=}\NormalTok{kc_houses}\OperatorTok{$}\NormalTok{grade, }\DataTypeTok{b=}\NormalTok{kc_houses}\OperatorTok{$}\NormalTok{price, }\DataTypeTok{lab_a =} \StringTok{"Grade"}\NormalTok{, }\DataTypeTok{lab_b =} \StringTok{"Price"}\NormalTok{)}
\end{Highlighting}
\end{Shaded}

\includegraphics{ClassPresentation_files/figure-latex/unnamed-chunk-11-1.pdf}

\hypertarget{scatterhistogramdensity-plots-sq.-ft.-above-vs-price}{%
\subsubsection{Scatter/Histogram/Density Plots: Sq. Ft. Above vs
Price}\label{scatterhistogramdensity-plots-sq.-ft.-above-vs-price}}

\begin{Shaded}
\begin{Highlighting}[]
\KeywordTok{ggplot_scatter_hist_combo}\NormalTok{(}\DataTypeTok{dat =}\NormalTok{ kc_houses, }\DataTypeTok{a=}\NormalTok{kc_houses}\OperatorTok{$}\NormalTok{sqft_above, }\DataTypeTok{b=}\NormalTok{kc_houses}\OperatorTok{$}\NormalTok{price, }\DataTypeTok{lab_a =} \StringTok{"Sq. Ft. Above"}\NormalTok{, }\DataTypeTok{lab_b =} \StringTok{"Price"}\NormalTok{)}
\end{Highlighting}
\end{Shaded}

\includegraphics{ClassPresentation_files/figure-latex/unnamed-chunk-12-1.pdf}

\hypertarget{scatterhistogramdensity-plots-sq.-ft-above-vs-sq.-ft.-living}{%
\subsubsection{Scatter/Histogram/Density Plots: Sq. Ft Above vs Sq. Ft.
Living}\label{scatterhistogramdensity-plots-sq.-ft-above-vs-sq.-ft.-living}}

\begin{Shaded}
\begin{Highlighting}[]
\KeywordTok{ggplot_scatter_hist_combo}\NormalTok{(}\DataTypeTok{dat =}\NormalTok{ kc_houses, }\DataTypeTok{a=}\NormalTok{kc_houses}\OperatorTok{$}\NormalTok{sqft_above, }\DataTypeTok{b=}\NormalTok{kc_houses}\OperatorTok{$}\NormalTok{sqft_living, }\DataTypeTok{lab_a =} \StringTok{"Sq. Ft. Above"}\NormalTok{, }\DataTypeTok{lab_b =} \StringTok{"Sq. Ft. Living"}\NormalTok{)}
\end{Highlighting}
\end{Shaded}

\includegraphics{ClassPresentation_files/figure-latex/unnamed-chunk-13-1.pdf}

\hypertarget{scatterhistogramdensity-plots-sq.-ft-above-vs-grade}{%
\subsubsection{Scatter/Histogram/Density Plots: Sq. Ft Above vs
Grade}\label{scatterhistogramdensity-plots-sq.-ft-above-vs-grade}}

\begin{Shaded}
\begin{Highlighting}[]
\KeywordTok{ggplot_scatter_hist_combo}\NormalTok{(}\DataTypeTok{dat =}\NormalTok{ kc_houses, }\DataTypeTok{a=}\NormalTok{kc_houses}\OperatorTok{$}\NormalTok{sqft_above, }\DataTypeTok{b=}\NormalTok{kc_houses}\OperatorTok{$}\NormalTok{grade, }\DataTypeTok{lab_a =} \StringTok{"Sq. Ft. Above"}\NormalTok{, }\DataTypeTok{lab_b =} \StringTok{"Grade"}\NormalTok{)}
\end{Highlighting}
\end{Shaded}

\includegraphics{ClassPresentation_files/figure-latex/unnamed-chunk-14-1.pdf}

\hypertarget{scatterhistogramdensity-plots-sq.-ft-living-vs-grade}{%
\subsubsection{Scatter/Histogram/Density Plots: Sq. Ft Living vs
Grade}\label{scatterhistogramdensity-plots-sq.-ft-living-vs-grade}}

\begin{Shaded}
\begin{Highlighting}[]
\KeywordTok{ggplot_scatter_hist_combo}\NormalTok{(}\DataTypeTok{dat =}\NormalTok{ kc_houses, }\DataTypeTok{a=}\NormalTok{kc_houses}\OperatorTok{$}\NormalTok{sqft_living, }\DataTypeTok{b=}\NormalTok{kc_houses}\OperatorTok{$}\NormalTok{grade, }\DataTypeTok{lab_a =} \StringTok{"Sq. Ft. Living"}\NormalTok{, }\DataTypeTok{lab_b =} \StringTok{"Grade"}\NormalTok{)}
\end{Highlighting}
\end{Shaded}

\includegraphics{ClassPresentation_files/figure-latex/unnamed-chunk-15-1.pdf}

\hypertarget{fitting-a-linear-regression-sq.-ft-living-vs-price}{%
\subsubsection{Fitting a linear regression: Sq. Ft Living vs
Price}\label{fitting-a-linear-regression-sq.-ft-living-vs-price}}

\begin{Shaded}
\begin{Highlighting}[]
\NormalTok{model<-}\KeywordTok{lm}\NormalTok{(price}\OperatorTok{~}\NormalTok{sqft_living, }\DataTypeTok{data =}\NormalTok{ kc_houses)}
\CommentTok{# Coefficent of Determination}
\KeywordTok{summary}\NormalTok{(model)}
\end{Highlighting}
\end{Shaded}

\begin{verbatim}
## 
## Call:
## lm(formula = price ~ sqft_living, data = kc_houses)
## 
## Residuals:
##      Min       1Q   Median       3Q      Max 
## -1478896  -147583   -24131   106274  4359590 
## 
## Coefficients:
##               Estimate Std. Error t value            Pr(>|t|)    
## (Intercept) -43988.892   4410.023  -9.975 <0.0000000000000002 ***
## sqft_living    280.863      1.939 144.819 <0.0000000000000002 ***
## ---
## Signif. codes:  0 '***' 0.001 '**' 0.01 '*' 0.05 '.' 0.1 ' ' 1
## 
## Residual standard error: 261700 on 21595 degrees of freedom
## Multiple R-squared:  0.4927, Adjusted R-squared:  0.4927 
## F-statistic: 2.097e+04 on 1 and 21595 DF,  p-value: < 0.00000000000000022
\end{verbatim}

\begin{Shaded}
\begin{Highlighting}[]
\CommentTok{# Analysis of Variance}
\KeywordTok{anova}\NormalTok{(model)}
\end{Highlighting}
\end{Shaded}

\begin{verbatim}
## Analysis of Variance Table
## 
## Response: price
##                Df           Sum Sq          Mean Sq F value
## sqft_living     1 1435979346411368 1435979346411368   20972
## Residuals   21595 1478602783996835      68469682056        
##                            Pr(>F)    
## sqft_living < 0.00000000000000022 ***
## Residuals                            
## ---
## Signif. codes:  0 '***' 0.001 '**' 0.01 '*' 0.05 '.' 0.1 ' ' 1
\end{verbatim}

\begin{Shaded}
\begin{Highlighting}[]
\CommentTok{# Variable Inflation Factor}
\CommentTok{#vif(model)}
\end{Highlighting}
\end{Shaded}

\hypertarget{fitting-a-linear-regression-grade-vs-price}{%
\subsubsection{Fitting a linear regression: Grade vs
Price}\label{fitting-a-linear-regression-grade-vs-price}}

\begin{Shaded}
\begin{Highlighting}[]
\NormalTok{model<-}\KeywordTok{lm}\NormalTok{(price}\OperatorTok{~}\NormalTok{grade, }\DataTypeTok{data =}\NormalTok{ kc_houses)}
\CommentTok{# Coefficent of Determination}
\KeywordTok{summary}\NormalTok{(model)}
\end{Highlighting}
\end{Shaded}

\begin{verbatim}
## 
## Call:
## lm(formula = price ~ grade, data = kc_houses)
## 
## Residuals:
##     Min      1Q  Median      3Q     Max 
## -819320 -151846  -36054   98154 6042365 
## 
## Coefficients:
##             Estimate Std. Error t value            Pr(>|t|)    
## (Intercept) -1061416      12286   -86.4 <0.0000000000000002 ***
## grade         209158       1586   131.9 <0.0000000000000002 ***
## ---
## Signif. codes:  0 '***' 0.001 '**' 0.01 '*' 0.05 '.' 0.1 ' ' 1
## 
## Residual standard error: 273400 on 21595 degrees of freedom
## Multiple R-squared:  0.4462, Adjusted R-squared:  0.4461 
## F-statistic: 1.74e+04 on 1 and 21595 DF,  p-value: < 0.00000000000000022
\end{verbatim}

\begin{Shaded}
\begin{Highlighting}[]
\CommentTok{# Analysis of Variance}
\KeywordTok{anova}\NormalTok{(model)}
\end{Highlighting}
\end{Shaded}

\begin{verbatim}
## Analysis of Variance Table
## 
## Response: price
##              Df           Sum Sq          Mean Sq F value
## grade         1 1300364813228575 1300364813228575   17396
## Residuals 21595 1614217317179626      74749586348        
##                          Pr(>F)    
## grade     < 0.00000000000000022 ***
## Residuals                          
## ---
## Signif. codes:  0 '***' 0.001 '**' 0.01 '*' 0.05 '.' 0.1 ' ' 1
\end{verbatim}

\begin{Shaded}
\begin{Highlighting}[]
\CommentTok{# Variable Inflation Factor}
\CommentTok{#vif(model)}
\end{Highlighting}
\end{Shaded}

\hypertarget{fitting-a-linear-regression-sq.-ft.-above-vs-price}{%
\subsubsection{Fitting a linear regression: Sq. Ft. Above vs
Price}\label{fitting-a-linear-regression-sq.-ft.-above-vs-price}}

\begin{Shaded}
\begin{Highlighting}[]
\NormalTok{model<-}\KeywordTok{lm}\NormalTok{(price}\OperatorTok{~}\NormalTok{sqft_above, }\DataTypeTok{data =}\NormalTok{ kc_houses)}
\CommentTok{# Coefficent of Determination}
\KeywordTok{summary}\NormalTok{(model)}
\end{Highlighting}
\end{Shaded}

\begin{verbatim}
## 
## Call:
## lm(formula = price ~ sqft_above, data = kc_houses)
## 
## Residuals:
##     Min      1Q  Median      3Q     Max 
## -914000 -165893  -41486  109326 5337755 
## 
## Coefficients:
##              Estimate Std. Error t value            Pr(>|t|)    
## (Intercept) 59757.111   4737.581   12.61 <0.0000000000000002 ***
## sqft_above    268.668      2.404  111.77 <0.0000000000000002 ***
## ---
## Signif. codes:  0 '***' 0.001 '**' 0.01 '*' 0.05 '.' 0.1 ' ' 1
## 
## Residual standard error: 292400 on 21595 degrees of freedom
## Multiple R-squared:  0.3665, Adjusted R-squared:  0.3664 
## F-statistic: 1.249e+04 on 1 and 21595 DF,  p-value: < 0.00000000000000022
\end{verbatim}

\begin{Shaded}
\begin{Highlighting}[]
\CommentTok{# Analysis of Variance}
\KeywordTok{anova}\NormalTok{(model)}
\end{Highlighting}
\end{Shaded}

\begin{verbatim}
## Analysis of Variance Table
## 
## Response: price
##               Df           Sum Sq          Mean Sq F value
## sqft_above     1 1068107925465501 1068107925465501   12492
## Residuals  21595 1846474204942702      85504709652        
##                           Pr(>F)    
## sqft_above < 0.00000000000000022 ***
## Residuals                           
## ---
## Signif. codes:  0 '***' 0.001 '**' 0.01 '*' 0.05 '.' 0.1 ' ' 1
\end{verbatim}

\begin{Shaded}
\begin{Highlighting}[]
\CommentTok{# Variable Inflation Factor}
\CommentTok{#vif(model)}
\end{Highlighting}
\end{Shaded}

\hypertarget{multiple-regression}{%
\subsection{Multiple Regression}\label{multiple-regression}}

\hypertarget{fitting-a-multiple-regression-sq.-ft.-living-grade-vs-price}{%
\subsubsection{Fitting a multiple regression: Sq. Ft. Living + Grade vs
Price}\label{fitting-a-multiple-regression-sq.-ft.-living-grade-vs-price}}

\begin{Shaded}
\begin{Highlighting}[]
\NormalTok{model<-}\KeywordTok{lm}\NormalTok{(price}\OperatorTok{~}\NormalTok{sqft_living}\OperatorTok{+}\NormalTok{grade, }\DataTypeTok{data =}\NormalTok{ kc_houses)}
\CommentTok{# Coefficent of Determination}
\KeywordTok{summary}\NormalTok{(model)}
\end{Highlighting}
\end{Shaded}

\begin{verbatim}
## 
## Call:
## lm(formula = price ~ sqft_living + grade, data = kc_houses)
## 
## Residuals:
##      Min       1Q   Median       3Q      Max 
## -1066199  -138328   -24864   100523  4793864 
## 
## Coefficients:
##                Estimate  Std. Error t value            Pr(>|t|)    
## (Intercept) -602791.788   13341.637  -45.18 <0.0000000000000002 ***
## sqft_living     184.121       2.872   64.10 <0.0000000000000002 ***
## grade         99251.094    2247.784   44.16 <0.0000000000000002 ***
## ---
## Signif. codes:  0 '***' 0.001 '**' 0.01 '*' 0.05 '.' 0.1 ' ' 1
## 
## Residual standard error: 250600 on 21594 degrees of freedom
## Multiple R-squared:  0.5347, Adjusted R-squared:  0.5347 
## F-statistic: 1.241e+04 on 2 and 21594 DF,  p-value: < 0.00000000000000022
\end{verbatim}

\begin{Shaded}
\begin{Highlighting}[]
\CommentTok{# Analysis of Variance}
\KeywordTok{anova}\NormalTok{(model)}
\end{Highlighting}
\end{Shaded}

\begin{verbatim}
## Analysis of Variance Table
## 
## Response: price
##                Df           Sum Sq          Mean Sq F value
## sqft_living     1 1435979346411368 1435979346411368 22865.0
## grade           1  122444337887608  122444337887608  1949.7
## Residuals   21594 1356158446109227      62802558401        
##                            Pr(>F)    
## sqft_living < 0.00000000000000022 ***
## grade       < 0.00000000000000022 ***
## Residuals                            
## ---
## Signif. codes:  0 '***' 0.001 '**' 0.01 '*' 0.05 '.' 0.1 ' ' 1
\end{verbatim}

\begin{Shaded}
\begin{Highlighting}[]
\CommentTok{# Variable Inflation Factor}
\KeywordTok{print}\NormalTok{(}\StringTok{'Variable Inflation Factor'}\NormalTok{)}
\end{Highlighting}
\end{Shaded}

\begin{verbatim}
## [1] "Variable Inflation Factor"
\end{verbatim}

\begin{Shaded}
\begin{Highlighting}[]
\KeywordTok{vif}\NormalTok{(model)}
\end{Highlighting}
\end{Shaded}

\begin{verbatim}
## sqft_living       grade 
##    2.391383    2.391383
\end{verbatim}

\hypertarget{fitting-a-multiple-regression-sq.-ft.-living-sq.-ft.-above-vs-price}{%
\subsubsection{Fitting a multiple regression: Sq. Ft. Living + Sq. Ft.
Above vs
Price}\label{fitting-a-multiple-regression-sq.-ft.-living-sq.-ft.-above-vs-price}}

\begin{Shaded}
\begin{Highlighting}[]
\NormalTok{model<-}\KeywordTok{lm}\NormalTok{(price}\OperatorTok{~}\NormalTok{sqft_living}\OperatorTok{+}\NormalTok{sqft_above, }\DataTypeTok{data =}\NormalTok{ kc_houses)}
\CommentTok{# Coefficent of Determination}
\KeywordTok{summary}\NormalTok{(model)}
\end{Highlighting}
\end{Shaded}

\begin{verbatim}
## 
## Call:
## lm(formula = price ~ sqft_living + sqft_above, data = kc_houses)
## 
## Residuals:
##      Min       1Q   Median       3Q      Max 
## -1505870  -147424   -23666   105379  4338962 
## 
## Coefficients:
##               Estimate Std. Error t value             Pr(>|t|)    
## (Intercept) -41266.657   4455.441  -9.262 < 0.0000000000000002 ***
## sqft_living    295.726      4.026  73.448 < 0.0000000000000002 ***
## sqft_above     -18.810      4.466  -4.212            0.0000254 ***
## ---
## Signif. codes:  0 '***' 0.001 '**' 0.01 '*' 0.05 '.' 0.1 ' ' 1
## 
## Residual standard error: 261600 on 21594 degrees of freedom
## Multiple R-squared:  0.4931, Adjusted R-squared:  0.4931 
## F-statistic: 1.05e+04 on 2 and 21594 DF,  p-value: < 0.00000000000000022
\end{verbatim}

\begin{Shaded}
\begin{Highlighting}[]
\CommentTok{# Analysis of Variance}
\KeywordTok{anova}\NormalTok{(model)}
\end{Highlighting}
\end{Shaded}

\begin{verbatim}
## Analysis of Variance Table
## 
## Response: price
##                Df           Sum Sq          Mean Sq   F value
## sqft_living     1 1435979346411368 1435979346411368 20988.743
## sqft_above      1    1213763626630    1213763626630    17.741
## Residuals   21594 1477389020370205      68416644455          
##                            Pr(>F)    
## sqft_living < 0.00000000000000022 ***
## sqft_above             0.00002542 ***
## Residuals                            
## ---
## Signif. codes:  0 '***' 0.001 '**' 0.01 '*' 0.05 '.' 0.1 ' ' 1
\end{verbatim}

\begin{Shaded}
\begin{Highlighting}[]
\CommentTok{# Variable Inflation Factor}
\KeywordTok{print}\NormalTok{(}\StringTok{'Variable Inflation Factor'}\NormalTok{)}
\end{Highlighting}
\end{Shaded}

\begin{verbatim}
## [1] "Variable Inflation Factor"
\end{verbatim}

\begin{Shaded}
\begin{Highlighting}[]
\KeywordTok{vif}\NormalTok{(model)}
\end{Highlighting}
\end{Shaded}

\begin{verbatim}
## sqft_living  sqft_above 
##    4.313332    4.313332
\end{verbatim}

\hypertarget{fitting-a-multiple-regression-grade-sq.-ft.-above-vs-price}{%
\subsubsection{Fitting a multiple regression: Grade + Sq. Ft. Above vs
Price}\label{fitting-a-multiple-regression-grade-sq.-ft.-above-vs-price}}

\begin{Shaded}
\begin{Highlighting}[]
\NormalTok{model<-}\KeywordTok{lm}\NormalTok{(price}\OperatorTok{~}\NormalTok{grade}\OperatorTok{+}\NormalTok{sqft_above, }\DataTypeTok{data =}\NormalTok{ kc_houses)}
\CommentTok{# Coefficent of Determination}
\KeywordTok{summary}\NormalTok{(model)}
\end{Highlighting}
\end{Shaded}

\begin{verbatim}
## 
## Call:
## lm(formula = price ~ grade + sqft_above, data = kc_houses)
## 
## Residuals:
##     Min      1Q  Median      3Q     Max 
## -797336 -149181  -33804  100261 5633593 
## 
## Coefficients:
##               Estimate Std. Error t value            Pr(>|t|)    
## (Intercept) -822646.15   14286.16  -57.58 <0.0000000000000002 ***
## grade        153695.00    2371.02   64.82 <0.0000000000000002 ***
## sqft_above      103.97       3.36   30.94 <0.0000000000000002 ***
## ---
## Signif. codes:  0 '***' 0.001 '**' 0.01 '*' 0.05 '.' 0.1 ' ' 1
## 
## Residual standard error: 267500 on 21594 degrees of freedom
## Multiple R-squared:  0.4697, Adjusted R-squared:  0.4696 
## F-statistic:  9562 on 2 and 21594 DF,  p-value: < 0.00000000000000022
\end{verbatim}

\begin{Shaded}
\begin{Highlighting}[]
\CommentTok{# Analysis of Variance}
\KeywordTok{anova}\NormalTok{(model)}
\end{Highlighting}
\end{Shaded}

\begin{verbatim}
## Analysis of Variance Table
## 
## Response: price
##               Df           Sum Sq          Mean Sq F value
## grade          1 1300364813228575 1300364813228575 18166.6
## sqft_above     1   68516724201419   68516724201419   957.2
## Residuals  21594 1545700592978208      71580095998        
##                           Pr(>F)    
## grade      < 0.00000000000000022 ***
## sqft_above < 0.00000000000000022 ***
## Residuals                           
## ---
## Signif. codes:  0 '***' 0.001 '**' 0.01 '*' 0.05 '.' 0.1 ' ' 1
\end{verbatim}

\begin{Shaded}
\begin{Highlighting}[]
\CommentTok{# Variable Inflation Factor}
\KeywordTok{print}\NormalTok{(}\StringTok{'Variable Inflation Factor'}\NormalTok{)}
\end{Highlighting}
\end{Shaded}

\begin{verbatim}
## [1] "Variable Inflation Factor"
\end{verbatim}

\begin{Shaded}
\begin{Highlighting}[]
\KeywordTok{vif}\NormalTok{(model)}
\end{Highlighting}
\end{Shaded}

\begin{verbatim}
##      grade sqft_above 
##   2.334516   2.334516
\end{verbatim}

\hypertarget{fitting-a-multiple-regression-sq.-ft.-living-grade-sq.-ft.-above-vs-price}{%
\subsubsection{Fitting a multiple regression: Sq. Ft. Living + Grade +
Sq. Ft. Above vs
Price}\label{fitting-a-multiple-regression-sq.-ft.-living-grade-sq.-ft.-above-vs-price}}

\begin{Shaded}
\begin{Highlighting}[]
\NormalTok{model<-}\KeywordTok{lm}\NormalTok{(price}\OperatorTok{~}\NormalTok{sqft_living}\OperatorTok{+}\NormalTok{grade}\OperatorTok{+}\NormalTok{sqft_above, }\DataTypeTok{data =}\NormalTok{ kc_houses)}
\CommentTok{# Coefficent of Determination}
\KeywordTok{summary}\NormalTok{(model)}
\end{Highlighting}
\end{Shaded}

\begin{verbatim}
## 
## Call:
## lm(formula = price ~ sqft_living + grade + sqft_above, data = kc_houses)
## 
## Residuals:
##      Min       1Q   Median       3Q      Max 
## -1120217  -136869   -24076    97184  4758690 
## 
## Coefficients:
##                Estimate  Std. Error t value            Pr(>|t|)    
## (Intercept) -656430.921   13591.272  -48.30 <0.0000000000000002 ***
## sqft_living     234.590       4.039   58.08 <0.0000000000000002 ***
## grade        110785.608    2325.608   47.64 <0.0000000000000002 ***
## sqft_above      -78.096       4.427  -17.64 <0.0000000000000002 ***
## ---
## Signif. codes:  0 '***' 0.001 '**' 0.01 '*' 0.05 '.' 0.1 ' ' 1
## 
## Residual standard error: 248800 on 21593 degrees of freedom
## Multiple R-squared:  0.5413, Adjusted R-squared:  0.5412 
## F-statistic:  8494 on 3 and 21593 DF,  p-value: < 0.00000000000000022
\end{verbatim}

\begin{Shaded}
\begin{Highlighting}[]
\CommentTok{# Analysis of Variance}
\KeywordTok{anova}\NormalTok{(model)}
\end{Highlighting}
\end{Shaded}

\begin{verbatim}
## Analysis of Variance Table
## 
## Response: price
##                Df           Sum Sq          Mean Sq  F value
## sqft_living     1 1435979346411368 1435979346411368 23193.47
## grade           1  122444337887608  122444337887608  1977.68
## sqft_above      1   19269340667109   19269340667109   311.23
## Residuals   21593 1336889105442118      61913078564         
##                            Pr(>F)    
## sqft_living < 0.00000000000000022 ***
## grade       < 0.00000000000000022 ***
## sqft_above  < 0.00000000000000022 ***
## Residuals                            
## ---
## Signif. codes:  0 '***' 0.001 '**' 0.01 '*' 0.05 '.' 0.1 ' ' 1
\end{verbatim}

\begin{Shaded}
\begin{Highlighting}[]
\CommentTok{# Variable Inflation Factor}
\KeywordTok{print}\NormalTok{(}\StringTok{'Variable Inflation Factor'}\NormalTok{)}
\end{Highlighting}
\end{Shaded}

\begin{verbatim}
## [1] "Variable Inflation Factor"
\end{verbatim}

\begin{Shaded}
\begin{Highlighting}[]
\KeywordTok{vif}\NormalTok{(model)}
\end{Highlighting}
\end{Shaded}

\begin{verbatim}
## sqft_living       grade  sqft_above 
##    4.797597    2.596616    4.683512
\end{verbatim}


\end{document}
